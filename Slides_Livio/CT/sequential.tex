\section{The sequential rejection principle}
\subsection{}
\begin{frame}
\frametitle{All similar methods}


\begin{enumerate}
\item Start testing hypotheses at some significance criterion
\item If any hypotheses rejected, set new significance criterion for unrejected hypotheses
\item Possibly new rejections
\item Stop if no new rejections occur
\end{enumerate}

\bigskip
\textcolor{cambridgedarkorange}{Are these all examples of the same procedure?}
\bigskip


\textcolor{cambridgedarkorange}{\textbf{The general procedure}}

\bigskip


$\mathcal{R}_{i}\subseteq \mathcal{H}$: the rejected hypotheses after step $i$

\begin{eqnarray*}
\mathcal{R}_{0} &=& \emptyset\\
\mathcal{R}_{i+1} &=& \mathcal{R}_{i} \cup \{H \in \mathcal{H}: S_{H} > c_H(\mathcal{R}_{i})\} 
\end{eqnarray*}

After every step the procedure adjusts the critical values on the basis of the new rejected set

\end{frame}
%%%%%%%%%%%%%%%%%%%%%%%%%%%%%%%%%%%%%%%%%%%%%%%%%%%%%%%%%%%%%%%%%%%%%%%%%%%%%%%%%%%%%%%%%%%%%%%%%%%%%%
\subsection{}
\begin{frame}
\frametitle{The Sequential Rejection Principle}



\textcolor{cambridgedarkorange}{\textbf{Theorem}\footnote{J.J. Goeman and A. Solari (2008) The sequential rejection principle of familywise error control. \emph{Submitted}}}

\bigskip
If a general sequentially rejective procedure fulfills two conditions
\begin{enumerate}
\item Monotonicity
\item Single step control
\end{enumerate}
Then it controls the FWE

\bigskip

\only<1>{
\textcolor{cambridgedarkorange}{Monotonicity condition:}
for every $\mathcal{S}\subseteq \mathcal{R}\subset\mathcal{H}$ and every $H\in \mathcal{H}\setminus\mathcal{R}$,

$$c_{H}(\mathcal{S})\geq c_{H}(\mathcal{R})$$

\textcolor{cambridgedarkorange}{In words:}
critical values of unrejected null hypotheses never increase with
more rejections

\textcolor{cambridgedarkorange}{Note:} monotonicity implies $c_{H}(\mathcal{R}_{i})\geq c_{H}(\mathcal{R}_{i+1})$, but this is too weak to guarantee FWE control
}
\only<2>{
\textcolor{cambridgedarkorange}{Single step condition:}
for every $\mathcal{R}\subset\mathcal{H}$ and $\mathcal{T}=\mathcal{H}\setminus \mathcal{R}$,

$$\Pr\left(\bigcup_{H\in \mathcal{T}} \left\{S_H > c_H(\mathcal{R})\right\}\right)\leq \alpha$$

\textcolor{cambridgedarkorange}{In words:}
FWE weak control is guaranteed at each single step.\\
It may be assumed that all previous rejections were correct
rejections
}


% \only<3>{
% \textcolor{cambridgedarkorange}{Proof:}
% \begin{itemize}
% \item `Worst case': all false null hypotheses have been rejected
% \item Single step condition guarantees FWE control in the worst case
% \item Monotonicity guarantees: no false rejections in the worst case implies no false rejections `on the way' to the worst case 
% \end{itemize}
% }

\only<4>{
\textcolor{cambridgedarkorange}{A unifying approach:}
\begin{itemize}
\item Facilitates formulation of FWE controlling procedures
\item Facilitates proof of FWE control
\item Makes connections between methods more obvious 
\end{itemize}
}

\end{frame}
