\subsection{Application: Neurotoxicity assay}
% \subsection{}
\begin{frame}
\frametitle{Neurotoxicity screening assay (FOB; Moser, 1989)}

\textcolor{cambridgedarkorange}{Goal:} Evaluation of neurotoxic effects of perchlorethylene \\

\bigskip

\textcolor{cambridgedarkorange}{Data:} The United States Enviromental Protection Agency published a guideline (FOB) to assess behavioural and neurologic functions in rats

\begin{itemize}
\item  \textcolor{cambridgedarkblue}{treatment} (1.5g/kg exposure level) versus  \textcolor{cambridgedarkblue}{control} (no exposure)
\item  \textcolor{cambridgedarkblue}{8 rats} at each group
\item  \textcolor{cambridgedarkblue}{21 endpoints} encompassing a wide spectrum of neurologic effects, grouped into  \textcolor{cambridgedarkblue}{6 domains}
\item at each endpoint, the response is \textcolor{cambridgedarkblue}{ordinal} on a scale from \\
1 (absence of adverse effect) to 4 (most severe reaction)
\end{itemize}


\bigskip
\textcolor{cambridgedarkorange}{Challenging statistical problem:} A large number of outcomes for a small sample of subjects

\end{frame}
%%%%%%%%%%%%%%%%%%%%%%%%%%%%%%%%%%%%%%%%%%%%%%%%%%%%%%%%%%%%%%%%%%%%%%%%%%%%%%%%%%%%%%%%%%%%%%%%%%%%%%
\subsection{}
\begin{frame}
\frametitle{Data}

\begin{table}[h!]
\footnotesize
\renewcommand{\tabcolsep}{0.5pc} % enlarge column spacing
\renewcommand{\arraystretch}{.6} % enlarge line spacing
\centering
\begin{tabular}{lll l cccc c cccc}
%\toprule
&Domain & & Endpoint & \multicolumn{9}{c}{Exposure (g/kg)}\\
       &&         && \multicolumn{4}{c}{0 (control)}   &&
       \multicolumn{4}{c}{1.5 (treatment)}\\
\cmidrule{5-8}\cmidrule{10-13}
          &             &
&& 1 & 2 & 3 & 4 && 1 & 2 & 3 & 4  \\
\cmidrule{5-8}\cmidrule{10-13}

&\textcolor{cambridgedarkblue}{Autonomic}

&& \textcolor{cambridgedarkorange}{Lacrimation} &   8   &   0   &   0   &  0  &&
                  5   &   0   &   3   &  0\\

&&&\textcolor{cambridgedarkorange}{Pupil } &        7   &   0   &   1   &  0  &&
                  5   &   0   &   3   &  0\\

&&&\textcolor{cambridgedarkorange}{Defecation} &    7   &   0   &   0   &  1  &&
                  7   &   1   &   0   &  0\\

&&&\textcolor{cambridgedarkorange}{Urination} &     4   &   3   &   1   &  0  &&
                  6   &   1   &   1   &  0\\
                  
                  \midrule

&\textcolor{cambridgedarkblue}{Sensorimotor}

&& \textcolor{cambridgedarkorange}{Approach}  &   8   &   0   &   0   &  0  &&
                 4   &   0   &   3   &  1\\

&&& \textcolor{cambridgedarkorange}{Click}  &       7   &   0   &   1   &  0  &&
                  7   &   0   &   1   &  0\\

&&& \textcolor{cambridgedarkorange}{Tail pinch}  &   6   &   0   &   2   &  0  &&
                   5   &   0   &   3   &  0\\

&&& \textcolor{cambridgedarkorange}{Touch}  &       8   &   0   &   0   &  0  &&
                  6   &   0   &   0   &  2\\

                  \midrule

&\textcolor{cambridgedarkblue}{CNS excitability}


&& \textcolor{cambridgedarkorange}{Handling}  &       6   &   2   &   0   &  0  &&
                     4   &   4   &   0   &  0\\

&&& \textcolor{cambridgedarkorange}{Clonic} &       4   &   4   &   0   &  0  &&
                  5   &   3   &   0   &  0\\

&&& \textcolor{cambridgedarkorange}{Arousal}  &     4   &   3   &   1   &  0  &&
                  3   &   0   &   3   &  2\\

&&& \textcolor{cambridgedarkorange}{Removal}  &       0   &   8   &   0   &  0  &&
                     0   &   7   &   1   &  0\\
                     
                                       \midrule
                     
&\textcolor{cambridgedarkblue}{CNS activity}

&& \textcolor{cambridgedarkorange}{Posture}  &       8   &   0   &   0   &  0  &&
                  7   &   0   &   1   &  0\\

&&& \textcolor{cambridgedarkorange}{Rearing} &      5   &   2   &   1   &  0  &&
                  4   &   2   &   2   &  0\\

                  \midrule

&\textcolor{cambridgedarkblue}{Neuromuscolar}

&& \textcolor{cambridgedarkorange}{Gait}&      8   &   0   &   0   &  0  &&
              3   &   5   &   0   &  0\\

&&& \textcolor{cambridgedarkorange}{Foot splay} &      6   &   1   &   1   &  0  &&
              6   &   1   &   1   &  0\\

&&& \textcolor{cambridgedarkorange}{Forelimb}  &      5   &   2   &   1   &  0  &&
              2   &   1   &   0   &  5\\

&&& \textcolor{cambridgedarkorange}{Hindlimb}  &      5   &   3   &   0   &  0  &&
              0   &   6   &   1   &  1\\

&&& \textcolor{cambridgedarkorange}{Righting}  &      8   &   0   &   0   &  0  &&
              5   &   2   &   1   &  0\\

                  \midrule

&\textcolor{cambridgedarkblue}{Psysicological}

&& \textcolor{cambridgedarkorange}{Weight} &      6   &   1   &   1   &  0  &&
              4   &   2   &   0   &  2\\

&&& \textcolor{cambridgedarkorange}{Temperature}  &      6   &   1   &   1   &  0  &&
                        4   &   3   &   0   &  1\\

%\bottomrule
\end{tabular}
\end{table}


\end{frame}
%%%%%%%%%%%%%%%%%%%%%%%%%%%%%%%%%%%%%%%%%%%%%%%%%%%%%%%%%%%%%%%%%%%%%%%%%%%%%%%%%%%%%%%%%%%%%%%%%%%%%%
\subsection{}
\begin{frame}
\frametitle{Multiple hypotheses}
$$H:  \textcolor{cambridgedarkorange}{Y_{H}} \stackrel{\mathrm{d}}{=} \textcolor{cambridgedarkblue}{X_{H}}\quad \mathrm{versus}\quad \tilde{H}:  \textcolor{cambridgedarkorange}{Y_{H}} \stackrel{\mathrm{st}}{\gneq} \textcolor{cambridgedarkblue}{X_{H}},\qquad H\in \mathcal{H}$$

\includegraphics[scale=.38]{endpoints}

\end{frame}
%%%%%%%%%%%%%%%%%%%%%%%%%%%%%%%%%%%%%%%%%%%%%%%%%%%%%%%%%%%%%%%%%%%%%%%%%%%%%%%%%%%%%%%%%%%%%%%%%%%%%%
% \subsection{}
% \begin{frame}
% \frametitle{Marginal stochastic order model}
% 
% 
% \begin{columns}[t]
% 
% \column{0.55\textwidth}
% 
% $\mathbf{X}=
% (X_A,X_B)\sim \mathrm{Multinomial}(\{\omega_{A,B}(i,j)\})$
% 
% \bigskip
% 
% \begin{tikzpicture}
% \foreach \x in {0,.5,1,1.5}
% \foreach \y in {0,.5,1,1.5}
% {
% \draw (\x,\y) +(-.25,-.25) rectangle ++(.25,.25);
% \draw[color=cambridgedarkorange]  (\x,-1) +(-.25,-.25) rectangle ++(.25,.25);
% \draw[color=cambridgedarkorange] (2.5,\y) +(-.25,-.25) rectangle ++(.25,.25);
% \draw (2.5,-1) +(-.25,-.25) rectangle ++(.25,.25);
% }
% \node at (2.5,2) {$\omega_A$};
% \node at (-.75,-1) {$\omega_B$};
% \node at (-.75,2) {$\omega_{A,B}$};
% \node at (2.5,-1) {$1$};
% \end{tikzpicture}
% 
% \column{0.55\textwidth}
% 
% $\mathbf{Y}=(Y_A,Y_B)\sim \mathrm{Multinomial}(\{\tau_{A,B}(i,j)\})$
% 
% \bigskip
% 
% \begin{tikzpicture}
% \foreach \x in {0,.5,1,1.5}
% \foreach \y in {0,.5,1,1.5}
% {
% \draw (\x,\y) +(-.25,-.25) rectangle ++(.25,.25);
% \draw[color=cambridgedarkorange]  (\x,-1) +(-.25,-.25) rectangle ++(.25,.25);
% \draw[color=cambridgedarkorange]  (2.5,\y) +(-.25,-.25) rectangle ++(.25,.25);
% \draw (2.5,-1) +(-.25,-.25) rectangle ++(.25,.25);
% }
% \node at (2.5,2) {$\tau_A$};
% \node at (-.75,-1) {$\tau_B$};
% \node at (-.75,2) {$\tau_{A,B}$};
% \node at (2.5,-1) {$1$};
% \end{tikzpicture}
% 
% \end{columns}
% 
% \bigskip
% 
% 
% $$\mathcal{M}_{\mathrm{M}}=\left\{ \begin{array}{cc}
%     X_{A}\stackrel{\mathrm{st}}{\leq}Y_{A}: &  \sum_{i=1}^{k}\omega_{A}(i)\geq \sum_{i=1}^{k}\tau_{A}(i), k=1,\ldots,4 \\ 
%     X_{B}\stackrel{\mathrm{st}}{\leq}Y_{B}: &  \sum_{j=1}^{l}\omega_{B}(j)\geq \sum_{j=1}^{k}\tau_{B}(j), l=1,\ldots,4 \\ 
%     \end{array}\right\}$$
% 
% \end{frame}
% %%%%%%%%%%%%%%%%%%%%%%%%%%%%%%%%%%%%%%%%%%%%%%%%%%%%%%%%%%%%%%%%%%%%%%%%%%%%%%%%%%%%%%%%%%%%%%%%%%%%%%
% \subsection{}
% \begin{frame}
% \frametitle{Marginal null model}
% 
% 
% \begin{columns}[t]
% 
% \column{0.55\textwidth}
% 
% $\mathbf{X}=(X_A,X_B)\sim \mathrm{Multinomial}(\{\omega_{A,B}(i,j)\})$
% 
% \bigskip
% 
% \begin{tikzpicture}
% \foreach \x in {0,.5,1,1.5}
% \foreach \y in {0,.5,1,1.5}
% {
% \draw (\x,\y) +(-.25,-.25) rectangle ++(.25,.25);
% \draw[color=cambridgedarkorange] (\x,-1) +(-.25,-.25) rectangle ++(.25,.25);
% \draw[color=cambridgedarkorange]   (2.5,\y) +(-.25,-.25) rectangle ++(.25,.25);
% \draw   (2.5,-1) +(-.25,-.25) rectangle ++(.25,.25);
% \draw (\x,\y) node{\tiny{$1/16$}};
% \draw[color=cambridgedarkorange]  (\x,-1) node{\tiny{$1/4$}};
% \draw[color=cambridgedarkorange]  (2.5,\y) node{\tiny{$1/4$}};
% }
% \node at (2.5,2) {$\omega_A$};
% \node at (-.75,-1) {$\omega_B$};
% \node at (-.75,2) {$\omega_{A,B}$};
% \node at (2.5,-1) {$1$};
% \end{tikzpicture}
% 
% \bigskip
% 
% $\qquad\lambda_{A}$: independence
% 
% \column{0.55\textwidth}
% 
% $\mathbf{Y}=(Y_A,Y_B)\sim \mathrm{Multinomial}(\{\tau_{A,B}(i,j)\})$
% 
% \bigskip
% 
% \begin{tikzpicture}
% \foreach \x in {0,.5,1,1.5}
% \foreach \y in {0,.5,1,1.5}
% {
% \draw (\x,\y) +(-.25,-.25) rectangle ++(.25,.25);
% \draw[color=cambridgedarkorange]   (\x,-1) +(-.25,-.25) rectangle ++(.25,.25);
% \draw[color=cambridgedarkorange]   (2.5,\y) +(-.25,-.25) rectangle ++(.25,.25);
% \draw (2.5,-1) +(-.25,-.25) rectangle ++(.25,.25);
% \draw[color=cambridgedarkorange]  (\x,-1) node{\tiny{$1/4$}};
% \draw[color=cambridgedarkorange]  (2.5,\y) node{\tiny{$1/4$}};
% \draw (1.5,0) node{\tiny{$1/4$}};
% \draw (1,.5) node{\tiny{$1/4$}};
% \draw (.5,1) node{\tiny{$1/4$}};
% \draw (0,1.5) node{\tiny{$1/4$}};
% }
% \node at (2.5,2) {$\tau_A$};
% \node at (-.75,-1) {$\tau_B$};
% \node at (-.75,2) {$\tau_{A,B}$};
% \node at (2.5,-1) {$1$};
% \end{tikzpicture}
% 
% \bigskip
% 
% $\quad\lambda_{B}$: complete dependence
% 
% \end{columns}
% 
% \bigskip
% 
% 
% $$\mathcal{M}_{\mathrm{M}_{0}}=\left\{ \begin{array}{cc}
%     X_{A}\stackrel{\mathrm{d}}{=}Y_{A}: &  \omega_{A}(i)=\tau_{A}(i), i=1,\ldots,4 \\ 
%     X_{B}\stackrel{\mathrm{d}}{=}Y_{B}: &  \omega_{B}(j)=\tau_{B}(j), j=1,\ldots,4 \\ 
% \end{array}\right\}$$
% 
% 
% 
% \end{frame}
% %%%%%%%%%%%%%%%%%%%%%%%%%%%%%%%%%%%%%%%%%%%%%%%%%%%%%%%%%%%%%%%%%%%%%%%%%%%%%%%%%%%%%%%%%%%%%%%%%%%%%%
% \begin{frame}
% \frametitle{Marginal exchangeability}
% 
% $\mathbf{X}_{r}=(X_{A,r},X_{B,r})^t,r=1,\ldots,8,\stackrel{\mathrm{i.i.d.}}\sim\mathbf{X}$  control sample\\
% $\mathbf{Y}_{r}=(Y_{A,r},Y_{B,r})^t,r=1,\ldots,8, \stackrel{\mathrm{i.i.d.}}\sim \mathbf{Y}$ treatment sample
% $(\mathbf{Z}_{1},\ldots,\mathbf{Z}_{16}) = (\mathbf{X}_{1},\ldots,\mathbf{X}_{8},\mathbf{Y}_{1},\ldots,\mathbf{Y}_{8})$ pooled sample
% 
% \bigskip
% 
% Under $\mathcal{M}_{\mathrm{M}_0}$,
% 
% $$(Z_{A,1},\ldots,Z_{A,16}) \stackrel{\mathrm{d}}{=}(Z_{A,\pi(1)},\ldots,Z_{A,\pi(16)})$$
% $$(Z_{B,1},\ldots,Z_{B,16})  \stackrel{\mathrm{d}}{=}( Z_{B,\pi'(1)},\ldots,Z_{B,\pi'(16)})$$
% 
% \bigskip
% 
% but in general
% 
% $$\left( \left(\begin{array}{c}
%     Z_{A,1} \\ 
%     Z_{B,1} \\ 
%   \end{array}\right),\ldots, \left(  \begin{array}{c}
%     Z_{A,16} \\ 
%     Z_{B,16} \\ 
%   \end{array}\right) \right)  \stackrel{\mathrm{d}}{\neq} \left(\left(\begin{array}{c}
%     Z_{A,\pi(1)} \\ 
%     Z_{B,\pi(1)} \\ 
%   \end{array}\right),\ldots, \left(  \begin{array}{c}
%     Z_{A,\pi(16)} \\ 
%     Z_{B,\pi(16)} \\ 
%   \end{array}\right)\right)$$
%  
% 
% %no information about the joint distribution of $\mathbf{Z}=
% %\left(  \begin{array}{c}
% %    Z_A \\ 
% %    Z_B \\ 
% %  \end{array}\right)$
% 
% %$\Pr(\mathbf{Z}_{1},\ldots,\mathbf{Z}_{16}) = \Pr(Z_{A,\pi(1)},\ldots,Z_{A,\pi(16)}) = 1/16!$
% 
% 
% 
% \end{frame}
% %%%%%%%%%%%%%%%%%%%%%%%%%%%%%%%%%%%%%%%%%%%%%%%%%%%%%%%%%%%%%%%%%%%%%%%%%%%%%%%%%%%%%%%%%%%%%%%%%%%%%%
% %%%%%%%%%%%%%%%%%%%%%%%%%%%%%%%%%%%%%%%%%%%%%%%%%%%%%%%%%%%%%%%%%%%%%%%%%%%%%%%%%%%%%%%%%%%%%%%%%%%%%%
% \subsection{}
% \begin{frame}
% \frametitle{Joint null model (no effect)}
% 
% 
% \begin{columns}[t]
% 
% \column{0.55\textwidth}
% 
% $\mathbf{X}=(X_A,X_B)\sim \mathrm{Multinomial}(\{\omega_{A,B}(i,j)\})$
% 
% \bigskip
% 
% \begin{tikzpicture}
% \foreach \x in {0,.5,1,1.5}
% \foreach \y in {0,.5,1,1.5}
% {
% \draw[color=cambridgedarkorange]   (\x,\y) +(-.25,-.25) rectangle ++(.25,.25);
% \draw (\x,-1) +(-.25,-.25) rectangle ++(.25,.25);
% \draw (2.5,\y) +(-.25,-.25) rectangle ++(.25,.25);
% \draw (2.5,-1) +(-.25,-.25) rectangle ++(.25,.25);
% \draw[color=cambridgedarkorange]   (\x,\y) node{\tiny{$\lambda$}};
% }
% \node at (2.5,2) {$\omega_A$};
% \node at (-.75,-1) {$\omega_B$};
% \node at (-.75,2) {$\omega_{A,B}$};
% \node at (2.5,-1) {$1$};
% \end{tikzpicture}
% 
% 
% 
% \column{0.55\textwidth}
% 
% $\mathbf{Y}=(Y_A,Y_B)\sim \mathrm{Multinomial}(\{\tau_{A,B}(i,j)\})$
% 
% \bigskip
% 
% \begin{tikzpicture}
% \foreach \x in {0,.5,1,1.5}
% \foreach \y in {0,.5,1,1.5}
% {
% \draw[color=cambridgedarkorange] (\x,\y) +(-.25,-.25) rectangle ++(.25,.25);
% \draw (\x,-1) +(-.25,-.25) rectangle ++(.25,.25);
% \draw (2.5,\y) +(-.25,-.25) rectangle ++(.25,.25);
% \draw (2.5,-1) +(-.25,-.25) rectangle ++(.25,.25);
% \draw[color=cambridgedarkorange] (\x,\y) node{\tiny{$\lambda$}};
% }
% \node at (2.5,2) {$\tau_A$};
% \node at (-.75,-1) {$\tau_B$};
% \node at (-.75,2) {$\tau_{A,B}$};
% \node at (2.5,-1) {$1$};
% \end{tikzpicture}
% 
% 
% 
% 
% \end{columns}
% 
% \bigskip
% 
% $\qquad\qquad\qquad\quad\quad\lambda$: any form of dependence
% 
% \bigskip
% 
% 
% $$\mathcal{M}_{\mathrm{J}_{0}}=\left\{ \begin{array}{cc}
%     \mathbf{X}\stackrel{\mathrm{d}}{=}\mathbf{Y}: &  \omega_{A,B}(i,j)=\tau_{A,B}(i,j) \,\, \forall\,(i,j)
% \end{array}\right\}\subset \mathcal{M}_{\mathrm{M}_{0}}$$
% 
% 
% 
% \end{frame}
% %%%%%%%%%%%%%%%%%%%%%%%%%%%%%%%%%%%%%%%%%%%%%%%%%%%%%%%%%%%%%%%%%%%%%%%%%%%%%%%%%%%%%%%%%%%%%%%%%%%%%%
% \begin{frame}
% \frametitle{Joint exchangeability}
% 
% 
% 
% Under $\mathcal{M}_{\mathrm{J}_0}$,
% 
% $$(\mathbf{Z}_{1},\ldots,\mathbf{Z}_{16}) \stackrel{\mathrm{d}}{=} (\mathbf{Z}_{\pi(1)},\ldots,\mathbf{Z}_{\pi(16)})$$
% 
% 
% \bigskip
% 
% the conditional critical region of $S_{\max}$ is similar
% 
% $$\Pr(S_{\max} > c|(\mathbf{z}_{1},\ldots,\mathbf{z}_{16}); \lambda) = \frac{\sum_{b=1}^{16!} I\{S_{\max}\circ \pi_b > c\}}{16!}\quad \forall\, \lambda\in \mathcal{M}_{\mathrm{J}_{0}}$$
% 
% \bigskip
% 
% single-step and monotonicity conditions fulfilled $\Rightarrow$ FWE control (exact)
% 
% 
% 
% \end{frame}
% %%%%%%%%%%%%%%%%%%%%%%%%%%%%%%%%%%%%%%%%%%%%%
% %%%%%%%%%%%%%%%%%%%%%%%%%%%%%%%%%%%%%%%%%%%%%
% \subsection{}
% \begin{frame}
% \frametitle{Joint stochastic order model}
% 
% 
% \begin{columns}[t]
% 
% \column{0.55\textwidth}
% 
% $\mathbf{X}=(X_A,X_B)\sim \mathrm{Multinomial}(\{\omega_{A,B}(i,j)\})$
% 
% \bigskip
% 
% \begin{tikzpicture}
% \foreach \x in {0,.5,1,1.5}
% \foreach \y in {0,.5,1,1.5}
% {
% \draw  (\x,\y) +(-.25,-.25) rectangle ++(.25,.25);
% \draw (\x,-1) +(-.25,-.25) rectangle ++(.25,.25);
% \draw (2.5,\y) +(-.25,-.25) rectangle ++(.25,.25);
% \draw (2.5,-1) +(-.25,-.25) rectangle ++(.25,.25);
% }
% \node at (2.5,2) {$\omega_A$};
% \node at (-.75,-1) {$\omega_B$};
% \node at (-.75,2) {$\omega_{A,B}$};
% \node at (2.5,-1) {$1$};
% \only<1>{\draw[color=cambridgedarkorange] (1,1.5) node{\tiny{$U$}};
% \draw[color=cambridgedarkorange] (1.5,1.5) node{\tiny{$U$}};
% \draw[color=cambridgedarkorange] (1.5,1) node{\tiny{$U$}};
% \draw[color=cambridgedarkorange] (1,1) node{\tiny{$U$}};
% }
% \only<2>{\draw[color=cambridgedarkorange] (1,1.5) node{\tiny{$U$}};
% \draw[color=cambridgedarkorange] (1.5,1.5) node{\tiny{$U$}};
% \draw[color=cambridgedarkorange] (1.5,1) node{\tiny{$U$}};
% \draw[color=cambridgedarkorange] (.5,1.5) node{\tiny{$U$}};
% \draw[color=cambridgedarkorange] (1.5,0.5) node{\tiny{$U$}};
% }
% \end{tikzpicture}
% 
% \column{0.55\textwidth}
% 
% $\mathbf{Y}=(Y_A,Y_B)\sim \mathrm{Multinomial}(\{\tau_{A,B}(i,j)\})$
% 
% \bigskip
% 
% \begin{tikzpicture}
% \foreach \x in {0,.5,1,1.5}
% \foreach \y in {0,.5,1,1.5}
% {
% \draw (\x,\y) +(-.25,-.25) rectangle ++(.25,.25);
% \draw (\x,-1) +(-.25,-.25) rectangle ++(.25,.25);
% \draw (2.5,\y) +(-.25,-.25) rectangle ++(.25,.25);
% \draw (2.5,-1) +(-.25,-.25) rectangle ++(.25,.25);
% }
% \node at (2.5,2) {$\tau_A$};
% \node at (-.75,-1) {$\tau_B$};
% \node at (-.75,2) {$\tau_{A,B}$};
% \node at (2.5,-1) {$1$};
% \only<1>{\draw[color=cambridgedarkorange] (1,1.5) node{\tiny{$U$}};
% \draw[color=cambridgedarkorange] (1.5,1.5) node{\tiny{$U$}};
% \draw[color=cambridgedarkorange] (1.5,1) node{\tiny{$U$}};
% \draw[color=cambridgedarkorange] (1,1) node{\tiny{$U$}};
% }
% \only<2>{\draw[color=cambridgedarkorange] (1,1.5) node{\tiny{$U$}};
% \draw[color=cambridgedarkorange] (1.5,1.5) node{\tiny{$U$}};
% \draw[color=cambridgedarkorange] (1.5,1) node{\tiny{$U$}};
% \draw[color=cambridgedarkorange] (.5,1.5) node{\tiny{$U$}};
% \draw[color=cambridgedarkorange] (1.5,0.5) node{\tiny{$U$}};
% }
% \end{tikzpicture}
% 
% \end{columns}
% 
% \bigskip
% 
% 
% $$\mathcal{M}_{\mathrm{J}}=\left\{ \begin{array}{cc}
%     \mathbf{X}\stackrel{\mathrm{st}}{\leq}\mathbf{Y}: &  \sum_{(i,j)\in U}\omega_{A,B}(i,j)\geq  \sum_{(i,j)\in U}\tau_{A,B}(i,j) \,\,\forall\,\,U\in \mathcal{U}
% \end{array}\right\}\subset \mathcal{M}_{\mathrm{M}}$$
% 
% \bigskip
% 
% $U\in \mathcal{U}$ upper set, $|\mathcal{U}|={8 \choose 4}=70$ 
% 
% \end{frame}
% %%%%%%%%%%%%%%%%%%%%%%%%%%%%%%%%
% \subsection{}
% \begin{frame}
% \frametitle{Model assumption}
% 
% \textcolor{cambridgedarkorange}{\textbf{Theorem}\footnote{A. Solari (2007) PhD Thesis; B. Klingenberg et al. (2008)  \emph{Biometrics}}}
% 
% \bigskip
% If the assumed model is $\mathcal{M}_{\mathrm{J}}$,
% 
% $$\mathcal{M}_{\mathrm{J}} \cap \mathcal{M}_{\mathrm{M_{0}}}= \mathcal{M}_{\mathrm{J_{0}}}$$
% 
% Then any permutation-based sequential procedure controls the FWE
% 
% 
% 
% 
% \end{frame}
%%%%%%%%%%%%%%%%%%%%%%%%%%%%%%%%%%%%%%%%%%%%%
%%%%%%%%%%%%%%%%%%%%%%%%%%%%%%%%%%%%%%%%%%%%%
\subsection{}
\begin{frame}
\frametitle{Test statistics}

Ordinal measurement: distances between categories are unknown

Mantel's test $S_H(\mathbf{w})$ depends on nondecreasing scores $\mathbf{w}$ assigned to categories. 

To overcome this problem, consider as the test statistic

$$S^{\max}_{H}=\max_{\mathbf{w}} \{S_{H}(\mathbf{w})\}$$

where $\mathbf{w}^{\max}$ maximizing $S_H(\mathbf{w})$ can be found by isotonic regression (adaptive test)

\begin{columns}[t]
\column{0.5\textwidth}
\bigskip
\begin{tabular}{c|cccc|c}
   Arousal  & I & II & III & IV & \\
 \hline
1.5 g/kg & 2 &  1 & 3 & 2 & 8\\
0 g/kg & 5 & 2 & 1 & 0 & 8 \\
\hline
\end{tabular}
% \bigskip
% 
\only<1> {$\mathbf{w}^{\mathrm{es}}=(0,1/3,2/3,1)$\\
\bigskip

$S_H(\mathbf{w}^{\mathrm{es}})=2.07$
}
\only<2>{$\mathbf{w}^{\max}=(0,0.07,0.65,1)$\\
\bigskip

$S_{H}^{\max}=S_H(\mathbf{w}^{\max})=2.15$
% Use $S_{H}^{\max}\circ\pi$ 
}
\column{0.5\textwidth}

\only<1>{\includegraphics[scale=0.3]{CTperE}}
\only<2>{\includegraphics[scale=0.3]{CTpmM}}


\end{columns}

\end{frame}
%%%%%%%%%%%%%%%%%%%%%%%%%%%%%%%%%%%%%%%%%%%%%%%%%%%%%%%%%%%%%%%%%%%%%%%%%%%%%%%%%%%%%%%%%%%%%%%%%%%%%%
\begin{frame}
\frametitle{Inference}

p-values for Endpoints (i.e. univariate test) are found by permutation approach (i.e. resampling based)

p-values for Domains (i.e. multivariate test) are found by combination of univariate ones (Pesarin 2001, klingenberg et al. 2009)


\end{frame}
%%%%%%%%%%%%%%%%%%%%%%%%%%%%%%%%%%%%%%%%%%%%%%%%%%%%%%%%%%%%%%%%%%%%%%%%%%%%%%%%%%%%%%%%%%%%%%%%%%%%%%
\subsection{}
\begin{frame}
\frametitle{Results: endpoint level}

\only<1>{\includegraphics[scale=0.43]{rawP}

\bigskip

$\odot$ = raw p-values 

}
\only<2>{\includegraphics[scale=0.43]{BonfHolmP}

\bigskip

adjusted p-values: $\odot$ = Inheritance, x = Holm
}

\end{frame}
%%%%%%%%%%%%%%%%%%%%%%%%%%%%%%%%%%%%%%%%%%%%%%%%%%%%%%%%%%%%%%%%%%%%%%%%%%%%%%%%%%%%%%%%%%%%%%%%%%%%%%
\begin{frame}
\frametitle{Results: domain level}

\includegraphics[scale=0.43]{Dom}

\bigskip 

adjusted p-values: $\bigcirc$ = Inheritance

\end{frame}
